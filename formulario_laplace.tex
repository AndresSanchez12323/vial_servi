\documentclass[9pt,a4paper,twocolumn]{article}
\usepackage[margin=0.4in]{geometry}
\usepackage{amsmath}
\usepackage{amssymb}
\setlength{\parindent}{0pt}
\setlength{\parskip}{1.5pt}
\setlength{\columnsep}{2pt}

\begin{document}
\scriptsize

\section*{FORMULARIO DE TRANSFORMADAS DE LAPLACE}

---

\subsection*{1. DEFINICIÓN DE LA TRANSFORMADA DE LAPLACE}

$\mathcal{L}\{f(t)\} = F(s) = \int_0^{\infty} e^{-st} f(t) \, dt$

$\mathcal{L}^{-1}\{F(s)\} = f(t)$

---

\subsection*{2. TRANSFORMADAS BÁSICAS (Teorema 7.2.1)}

$\mathcal{L}\{1\} = \frac{1}{s}, \quad s > 0$

$\mathcal{L}\{t^n\} = \frac{n!}{s^{n+1}}, \quad n = 0, 1, 2, \ldots$

$\mathcal{L}\{t\} = \frac{1}{s^2}$

$\mathcal{L}\{t^2\} = \frac{2}{s^3}$

$\mathcal{L}\{t^3\} = \frac{6}{s^4}$

$\mathcal{L}\{t^4\} = \frac{24}{s^5}$

$\mathcal{L}\{e^{at}\} = \frac{1}{s-a}, \quad s > a$

$\mathcal{L}\{\sin(kt)\} = \frac{k}{s^2 + k^2}$

$\mathcal{L}\{\cos(kt)\} = \frac{s}{s^2 + k^2}$

$\mathcal{L}\{\sinh(kt)\} = \frac{k}{s^2 - k^2}$

$\mathcal{L}\{\cosh(kt)\} = \frac{s}{s^2 - k^2}$

---

\subsection*{3. TRANSFORMADAS INVERSAS BÁSICAS}

$\mathcal{L}^{-1}\left\{\frac{1}{s}\right\} = 1$

$\mathcal{L}^{-1}\left\{\frac{1}{s^2}\right\} = t$

$\mathcal{L}^{-1}\left\{\frac{1}{s^3}\right\} = \frac{t^2}{2}$

$\mathcal{L}^{-1}\left\{\frac{1}{s^4}\right\} = \frac{t^3}{6}$

$\mathcal{L}^{-1}\left\{\frac{1}{s^5}\right\} = \frac{t^4}{24}$

$\mathcal{L}^{-1}\left\{\frac{1}{s^{n+1}}\right\} = \frac{t^n}{n!}$

$\mathcal{L}^{-1}\left\{\frac{1}{s-a}\right\} = e^{at}$

$\mathcal{L}^{-1}\left\{\frac{k}{s^2+k^2}\right\} = \sin(kt)$

$\mathcal{L}^{-1}\left\{\frac{s}{s^2+k^2}\right\} = \cos(kt)$

$\mathcal{L}^{-1}\left\{\frac{k}{s^2-k^2}\right\} = \sinh(kt)$

$\mathcal{L}^{-1}\left\{\frac{s}{s^2-k^2}\right\} = \cosh(kt)$

---

\subsection*{4. PROPIEDADES DE LINEALIDAD}

$\mathcal{L}\{af(t) + bg(t)\} = a\mathcal{L}\{f(t)\} + b\mathcal{L}\{g(t)\}$

$\mathcal{L}^{-1}\{aF(s) + bG(s)\} = a\mathcal{L}^{-1}\{F(s)\} + b\mathcal{L}^{-1}\{G(s)\}$

---

\subsection*{5. PRIMER TEOREMA DE TRASLACIÓN}

Si $\mathcal{L}\{f(t)\} = F(s)$, entonces:

$\boxed{\mathcal{L}\{e^{at}f(t)\} = F(s-a)}$

$\boxed{\mathcal{L}^{-1}\{F(s-a)\} = e^{at}f(t)}$

\textbf{Aplicaciones importantes:}

$\mathcal{L}\{e^{at}t^n\} = \frac{n!}{(s-a)^{n+1}}$

$\mathcal{L}\{e^{at}\sin(kt)\} = \frac{k}{(s-a)^2 + k^2}$

$\mathcal{L}\{e^{at}\cos(kt)\} = \frac{s-a}{(s-a)^2 + k^2}$

$\mathcal{L}^{-1}\left\{\frac{1}{(s-a)^{n+1}}\right\} = \frac{t^n e^{at}}{n!}$

$\mathcal{L}^{-1}\left\{\frac{k}{(s-a)^2+k^2}\right\} = e^{at}\sin(kt)$

$\mathcal{L}^{-1}\left\{\frac{s-a}{(s-a)^2+k^2}\right\} = e^{at}\cos(kt)$

---

\subsection*{6. TRANSFORMADAS DE DERIVADAS}

\textbf{Primera derivada:}
$\boxed{\mathcal{L}\{y'\} = sY(s) - y(0)}$

\textbf{Segunda derivada:}
$\boxed{\mathcal{L}\{y''\} = s^2Y(s) - sy(0) - y'(0)}$

\textbf{Tercera derivada:}
$\boxed{\mathcal{L}\{y'''\} = s^3Y(s) - s^2y(0) - sy'(0) - y''(0)}$

\textbf{n-ésima derivada:}
$\mathcal{L}\{y^{(n)}\} = s^nY(s) - s^{n-1}y(0) - s^{n-2}y'(0) - \cdots - y^{(n-1)}(0)$

---

\subsection*{7. MÉTODO PARA RESOLVER PVI CON LAPLACE}

\textbf{Paso 1:} Aplicar $\mathcal{L}$ a ambos lados de la EDO

\textbf{Paso 2:} Usar las fórmulas de derivadas:
\begin{itemize}
\item $\mathcal{L}\{y'\} = sY(s) - y(0)$
\item $\mathcal{L}\{y''\} = s^2Y(s) - sy(0) - y'(0)$
\end{itemize}

\textbf{Paso 3:} Sustituir condiciones iniciales

\textbf{Paso 4:} Despejar $Y(s)$

\textbf{Paso 5:} Descomponer en fracciones parciales

\textbf{Paso 6:} Aplicar $\mathcal{L}^{-1}$ para obtener $y(t)$

---

\subsection*{8. FRACCIONES PARCIALES}

\textbf{Factores lineales distintos:}
$\frac{P(s)}{(s-a)(s-b)(s-c)} = \frac{A}{s-a} + \frac{B}{s-b} + \frac{C}{s-c}$

\textbf{Factores lineales repetidos:}
$\frac{P(s)}{(s-a)^3} = \frac{A}{s-a} + \frac{B}{(s-a)^2} + \frac{C}{(s-a)^3}$

\textbf{Factores cuadráticos:}
$\frac{P(s)}{(s^2+k^2)} = \frac{As+B}{s^2+k^2}$

\textbf{Factores cuadráticos repetidos:}
$\frac{P(s)}{(s^2+k^2)^2} = \frac{As+B}{s^2+k^2} + \frac{Cs+D}{(s^2+k^2)^2}$

\textbf{Combinación de factores:}
$\frac{P(s)}{(s-a)(s^2+k^2)} = \frac{A}{s-a} + \frac{Bs+C}{s^2+k^2}$

---

\subsection*{9. MÉTODO DE COVER-UP (Heaviside)}

Para $\frac{P(s)}{(s-a)(s-b)(s-c)}$:

$A = \frac{P(a)}{(a-b)(a-c)}$ (evaluar P(s) en $s=a$, omitiendo $(s-a)$)

$B = \frac{P(b)}{(b-a)(b-c)}$

$C = \frac{P(c)}{(c-a)(c-b)}$

---

\subsection*{10. MANIPULACIONES ALGEBRAICAS ÚTILES}

\textbf{Completar cuadrados:}
$s^2 + 2as + b = (s+a)^2 + (b-a^2)$

\textbf{Separar fracciones:}
$\frac{2s-6}{s^2+9} = \frac{2s}{s^2+9} - \frac{6}{s^2+9}$

$= 2\cdot\frac{s}{s^2+9} - 2\cdot\frac{3}{s^2+9}$

\textbf{Ajustar al formato estándar:}
$\frac{1}{4s+1} = \frac{1}{4(s+\frac{1}{4})} = \frac{1}{4}\cdot\frac{1}{s+\frac{1}{4}}$

$\frac{1}{5s-2} = \frac{1}{5(s-\frac{2}{5})} = \frac{1}{5}\cdot\frac{1}{s-\frac{2}{5}}$

$\frac{1}{4s^2+1} = \frac{1}{4(s^2+\frac{1}{4})} = \frac{1}{4}\cdot\frac{1}{s^2+(\frac{1}{2})^2}$

\textbf{Expandir polinomios:}
$(s+1)^3 = s^3 + 3s^2 + 3s + 1$

$(s+2)^2 = s^2 + 4s + 4$

---

\subsection*{11. TRANSFORMADAS DE FUNCIONES ESPECIALES}

$\mathcal{L}\{t \sin(kt)\} = \frac{2ks}{(s^2+k^2)^2}$

$\mathcal{L}\{t \cos(kt)\} = \frac{s^2-k^2}{(s^2+k^2)^2}$

$\mathcal{L}\{e^{at}t^n\} = \frac{n!}{(s-a)^{n+1}}$

$\mathcal{L}\{\sin(kt) - kt\cos(kt)\} = \frac{2k^3}{(s^2+k^2)^2}$

---

\subsection*{12. FACTORIZACIÓN DE DENOMINADORES}

$s^2 - a^2 = (s-a)(s+a)$

$s^2 + 2s - 3 = (s+3)(s-1)$

$s^2 + s - 20 = (s+5)(s-4)$

$s^4 - 9 = (s^2-3)(s^2+3) = (s-\sqrt{3})(s+\sqrt{3})(s^2+3)$

$s^4 + 5s^2 + 4 = (s^2+1)(s^2+4)$

$s^3 + 5s = s(s^2+5)$

$s^2 + 3s = s(s+3)$

$s^2 - 4s = s(s-4)$

---

\subsection*{13. IDENTIDADES TRIGONOMÉTRICAS E HIPERBÓLICAS}

$\sinh(t) = \frac{e^t - e^{-t}}{2}$

$\cosh(t) = \frac{e^t + e^{-t}}{2}$

$\sin^2(t) + \cos^2(t) = 1$

$\cosh^2(t) - \sinh^2(t) = 1$

---

\subsection*{14. RESUMEN DE INVERTAS FRECUENTES}

$\mathcal{L}^{-1}\left\{\frac{1}{s(s-a)}\right\} = \frac{1}{a}(e^{at}-1)$

$\mathcal{L}^{-1}\left\{\frac{1}{(s-a)(s-b)}\right\} = \frac{e^{at}-e^{bt}}{a-b}$

$\mathcal{L}^{-1}\left\{\frac{s}{(s-a)(s-b)}\right\} = \frac{ae^{at}-be^{bt}}{a-b}$

$\mathcal{L}^{-1}\left\{\frac{1}{s(s^2+k^2)}\right\} = \frac{1}{k^2}(1-\cos(kt))$

$\mathcal{L}^{-1}\left\{\frac{1}{s^2(s^2+k^2)}\right\} = \frac{1}{k^3}(kt-\sin(kt))$

$\mathcal{L}^{-1}\left\{\frac{s}{(s^2+a^2)(s^2+b^2)}\right\} = \frac{\cos(at)-\cos(bt)}{b^2-a^2}$

$\mathcal{L}^{-1}\left\{\frac{1}{(s^2+a^2)(s^2+b^2)}\right\} = \frac{b\sin(at)-a\sin(bt)}{ab(a^2-b^2)}$

---

\subsection*{15. EJEMPLOS TIPO EXAMEN}

\textbf{Tipo 1:} $\mathcal{L}^{-1}\left\{\frac{1}{s^n}\right\}$

Usar: $\mathcal{L}^{-1}\left\{\frac{1}{s^{n}}\right\} = \frac{t^{n-1}}{(n-1)!}$

\textbf{Tipo 2:} $\mathcal{L}^{-1}\left\{\frac{1}{as+b}\right\}$

Reescribir: $\frac{1}{a(s+\frac{b}{a})} = \frac{1}{a}e^{-\frac{b}{a}t}$

\textbf{Tipo 3:} $\mathcal{L}^{-1}\left\{\frac{As+B}{s^2+k^2}\right\}$

Separar: $A\cdot\frac{s}{s^2+k^2} + \frac{B}{k}\cdot\frac{k}{s^2+k^2}$

Resultado: $A\cos(kt) + \frac{B}{k}\sin(kt)$

\textbf{Tipo 4:} Fracciones parciales

Factorizar denominador $\to$ descomponer $\to$ aplicar $\mathcal{L}^{-1}$

\textbf{Tipo 5:} PVI con Laplace

EDO $\xrightarrow{\mathcal{L}}$ ec. algebraica $\to$ despejar $Y(s)$ $\to$ fracciones parciales $\xrightarrow{\mathcal{L}^{-1}}$ $y(t)$

---

\end{document}
